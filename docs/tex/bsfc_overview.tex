\documentclass{beamer}

\usetheme{Boadilla}
\usecolortheme{beaver}

\usepackage{physics}
\usepackage{units}

\begin{document}


\title[BSFC]{Optimal Bayesian Spectral Fitting of Near-Gaussian Lines with Gauss-Hermite Functions}

\author[N.M. Cao and F. Sciortino]{Norman M. Cao, Francesco Sciortino}
\date{April xxth, 2018}

\begin{frame}
	\titlepage
\end{frame}

\section{Math}

\begin{frame}
	\frametitle{Background and Overview}
	HIREX and such
\end{frame}

\begin{frame}
	\frametitle{Gauss-Hermite Functions}
	The (Probabilist's) Hermite polynomials are an orthogonal set of polynomials under a Gaussian weight function, e.g. satisfying
	\begin{equation}
	\int_{-\infty}^{\infty} He_m(x) He_n(x) e^{-x^2/2} \dd{x} \equiv \left< He_m, He_n \right> = \sqrt{2 \pi} n! \delta_{nm}
	\end{equation}
	The first few and their plots of the Hermite function \(He_n(x) e^{-x^2/2}\) are shown below
	\begin{equation}
	\begin{gathered}
	He_0(x) = 1 \\
	He_1(x) = x \\
	He_2(x) = x^2-1
	\end{gathered}
	\end{equation}
	Note that for example, the 0th Hermite function is simply a Gaussian.
\end{frame}

\begin{frame}
	\frametitle{Gauss-Hermite Functions and Moments}
	Given a function \(h(x) \exp(-x^2/2) = \left[\sum_j a_j He_j(x)\right] \exp(-x^2/2)\), the first few (unnormalized) moments are very easy to calculate:
	
	\begin{equation}
	\begin{gathered}
	M_0 = \int_{-\infty}^{\infty} 1 \cdot \sum_j a_j He_j(x) \exp(-x^2/2) = \left< 1, h \right> = \sqrt{2\pi} a_0 \\
	M_1 = \left< x, h \right> = \sqrt{2 \pi}a_1 \\
	M_2 = \left< x^2, h \right> = A a_0 + B a_2
	\end{gathered}
	\end{equation}
	
	Note that the higher order Hermite functions are not involved with the moment calculations at all
\end{frame}

\begin{frame}
	\frametitle{Chi-Squared Minimization of Near-Gaussian Lines}
	Fitting of spectral lines typically involves minimizing \(\chi^2\), which from the Bayesian perspective, is finding the maximum a posteriori (MAP) estimate of a given set of photon counts
	\begin{equation}
	\hat{\theta} = \underset{\theta}{\operatorname{argmin}} \left[\sum_{i} \frac{(f_i(\theta) - N_i)^2}{N_i}\right] \equiv \underset{\theta}{\operatorname{argmin}} \left[\chi^2(\theta, \{N_i\})\right]
	\end{equation}
	If \(\mu(N_i) = C + (A_0+A(x_i))e^{-x_i^2/2} \), \(f_i = \hat{C} + h(\theta, x_i) e^{-x_i^2/2}\), where \(x_i \equiv (\lambda_i - \lambda_c)/s\) and \(A_0\gg C, A\), then
	\begin{align}
	E[\chi^2] &\approx \sum_{i} \frac{(h(\theta, x_i) - (A_0 + A(x_i))^2 e^{-x^2}}{A_0 e^{-x^2/2}} \\
	&\approx \sum_{i} D \left[(h(\theta, x_i) - (A_0 + A(x_i)))^2 e^{-x^2/2}\right] \\
	&\approx \left<h(\theta) - (A_0 + A), h(\theta)-(A_0 + A) \right>
	\end{align}
\end{frame}

\begin{frame}
	\frametitle{Orthogonal Projection onto Hermite Basis}
	\begin{equation}
	E[\chi^2] \approx \left<h(\theta) - (A_0 + A), h(\theta)-(A_0 + A) \right>
	\end{equation}
	\begin{itemize}
		\item This is a least squares minimization with a generalized inner product. The Hermite polynomials form an orthogonal basis for this inner product.
		\item If \(h(\theta)\) is a sum of Hermite polynomials, then the solution is given by an orthogonal projection of \(A_0 + A\) onto its component Hermite polynomials.
		\item For calculating moments, it doesn't even matter if convergence is bad! The polynomials are orthogonal!
	\end{itemize}
\end{frame}

\begin{frame}
	\frametitle{Fitting Actual Data}
	\begin{itemize}
		\item To extend to multiple lines, approximate the sum over multiple regions - multiple approximately orthogonal Hermite basis sets
		\item Need to find \(x_i = (\lambda_i - \lambda_0)/s\) - assume the lowest order center and scale parameters are the same for all lines
		\item Perform nonlinear optimization
	\end{itemize}
\end{frame}

\begin{frame}
	\frametitle{Analytic Calculation of Error -- TODO}
	Approximating the output error as Gaussian (see the corner plots), can perform propagation of error:
	\begin{equation}
	\Sigma^\theta = J \Sigma^N J^T
	\end{equation}
	How to calculate this though, it's a nonlinear optimization? Key is that for fixed \(\lambda_0, s\) then the Hermite polynomial coefficients are the result of an orthogonal projection of the vector \(N\) (i.e. a linear operation), so that part is easy. To calculate \(\pdv{s}{N}\), note that \(\theta(\lambda_0,s, \{a_j\})\) calculate the pullback of \(ds\) onto \(V\) and then project \(N\) onto the resulting vector space.
\end{frame}


\end{document}
